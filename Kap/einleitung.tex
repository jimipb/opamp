\chapter{Einleitung}
In der modernen experimentellen Physik haben nahezu alle Messtechniken, egal ob in der Teilchen-, Atom-, oder Festkörperphysik, eine Gemeinsamkeit: die im Experiment untersuchten Effekte werden umgewandelt in Spannungs- oder Stromsignale. Eine weitere Gemeinsamkeit ist die digitale Erfassung und Weiterverarbeitung der Messwerte mit Computerunterstützung. Die Schnittstellen zwischen der „analogen“ Welt und der digitalen Datenerfassung und Verarbeitung sind typischerweise Verstärkersysteme, in denen die Messsignale aufbereitet werden, und Analog-Digital-Wandlerkarten. Im Rahmen des Praktikums soll anhand eines Beispiels gezeigt werden, wie mithilfe von Operationsverstärkern und Frequenzfiltern reale Messsignale so konditioniert werden können, dass sie mit einem A/D-Wandler aufgenommen werden können. Als Signalquelle dienen dabei die Versuchsteilnehmer selbst: die schwachen elektrischen Signale, die den Herzschlag eines Menschen begleiten, werden mittels Elektroden an den Armen abgegriffen. Ziel des Praktikumsversuchs ist der schrittweise Aufbau eines Verstärkersystems zur Messung eines Elektrokardiogramms.

